Model-driven engineering (MDE) has traditionally adopted a strict hierarchical two-level approach, where metamodels reside in a certain meta-level, and models are created one meta-level below by using types from the metamodel. Notable examples of this approach are the widespread Eclipse Modeling Framework (EMF)~\citep{emf} and the Object Management Group Meta-Object Facility(MOF)~\citep{omg2013mof}.

This strict approach shows its limitations when modeling complex domains requiring more than one level of specialization, e.g., to adapt the model to application sub domains. Indeed, the two-level approach fails to acknowledge and support: 1) the existence of different forms of classification and/or instantiation (e.g., ontological vs linguistic); and 2) the duality type-object for model elements. Therefore, while it may still be used for the aforementioned complex modeling tasks, the strict approach entails the introduction of accidental complexity in both the modeling process and the resulting modeling artefacts. 

In order to tackle this problem, the \emph{multi-level} modeling paradigm have been introduced. Contrary to the strict approach, multi-level modeling advocates the use of a flexible number of levels as well as a flexibilization of the relations between them. This paradigm is gaining traction lately within the modeling community as evidenced by the contribution of many new multi-level modeling approaches and tools such as Melanee~\citep{melanee}, MetaDepth~\citep{metadepth}, MultEcore~\citep{multecore2016}, DeepTelos\citep{deeptelos2016} or DMLA~\citep{dmla2017}. Taking advantage of this vibrant state of affairs, and in order to foster discussion and enable comparison between competing approaches, a multi-level modeling challenge has been created. This paper is a response to the latest multi-level modeling challenge
which consists in providing a solution to the problem of specifying and enacting processes. Solutions to the \emph{Multi-level Process Challenge} must fulfill a number of requirements for a process representation defined at an abstract process-definition level and at various more concrete domain-specific levels, resulting in a multi-level hierarchy of related models. We provide a solution based on model federation~\citep{Golra2016-federation}.

Model federation is a multi model management approach based on the use of virtual models and loosely coupled links. The models in a federation remain autonomous and represented in their original technological spaces whereas virtual models (also called conceptual models) and links serve as control components used to present different views to the users and maintain synchronization. 
Our solution is based on the model federation infrastructure, namely, the use of virtual models and links. Concretely, we use OpenFlexo and its internal Flexo Modeling Language (FML), a language to create, link and manage virtual models, in order to solve the MULTI process challenge while the federation aspect is used solely so as to provide tooling for the resulting process language. Our solution, which is fully implemented and executable, meets all the requirements. 
%Links among models can be within a same level of abstraction or across levels of abstractions which gives, 

%as our answer to the challenge shows, a great flexibility.

The rest of the paper is organized as follows. Section~\ref{sec:technology} presents our approach and the technology it relies on. Section~\ref{sec:analysis} is the analysis of the problem. We detail our model in section~\ref{sec:model} and show it satisfies the challenge requirements in section~\ref{sec:requirements}. We assess the modeling solution in section~\ref{sec:discussion}. Section~\ref{sec:relatedwork} presents the related work before the conclusion.%~\ref{sec:conclusions}.


% \cite{multiProcessChallenge2019}

% \emph{Submissions responding to the challenge should describe
% a multi-level model conforming to the case description,
% including justifications for non-trivial design decisions. In
% order to foster comparability between solutions, respondents
% are asked to make sure that concepts of the case description
% are explicitly represented by one or more model elements.}\\

% \emph{criteria:
% \begin{enumerate}
%     \item Does the response address the established domain as described in the challenge description and demonstrate the use of modeling features?
%     \item Does it evaluate/discuss its proposed modeling solution against the criteria described in the challenge description?
%     \item Does it discuss the merits and limitations of the employed modeling technique?
%     Authors are invited to suggest further requirements to demonstrate particular benefits of the approaches they adopt.
% \end{enumerate}
% }

% structure : plus ou moins imposée par le challenge (structure actuelle = sections au moins attendues) 