Model-driven engineering (MDE) has traditionally adopted a strict hierarchical two-level approach, where metamodels reside in a certain meta-level, and models are created one meta-level below by using types from the metamodel. Notable examples of this approach are the widespread Eclipse Modeling Framework (EMF)~\citep{emf} and the Object Management Group Meta-Object Facility(MOF)~\citep{omg2013mof}.

This strict approach shows its limitations when modeling complex domains requiring more than one level of specialization, e.g., to adapt the model to application sub domains. Indeed, the two-level approach fails to acknowledge and support: 1) the existence of different forms of classification and/or instantiation (e.g., ontological vs linguistic); and 2) the duality type-object for model elements. Therefore, while it may still be used for the aforementioned complex modeling tasks, the strict approach entails the introduction of accidental complexity in both the modeling process and the resulting modeling artefacts. 

In order to tackle this problem, the \emph{multi-level} modeling paradigm have been introduced. Contrary to the strict approach, multi-level modeling advocates the use of a flexible number of levels as well as a flexibilization of the relations between levels.

This approach is gaining interest lately within the modeling community as shown by the MULTI level-modeling workshop and the daghtul seminar~\citep{almeida2018multi} and number of tools. Taking advantage of this vibrant state of affairs, and in order to foster discussion and enable comparison between competing approaches, a multi-level modeling challenge has been created. This paper is a response to the latest multi-level modeling challenge which consist of modeling processes. We do so by using model federation and the OpenFlexo framework.

Model federation is a modeling approach in which loosely couple reified models. Links among models can be within a same level of abstraction or across levels of abstractions which gives, as our answer to the challenge shows, a great flexibility.

Our solution, which is fully implemented and executable, meets all the requirements.

The rest of the paper is organized as follows.


% \cite{multiProcessChallenge2019}

% \emph{Submissions responding to the challenge should describe
% a multi-level model conforming to the case description,
% including justifications for non-trivial design decisions. In
% order to foster comparability between solutions, respondents
% are asked to make sure that concepts of the case description
% are explicitly represented by one or more model elements.}\\

% \emph{criteria:
% \begin{enumerate}
%     \item Does the response address the established domain as described in the challenge description and demonstrate the use of modeling features?
%     \item Does it evaluate/discuss its proposed modeling solution against the criteria described in the challenge description?
%     \item Does it discuss the merits and limitations of the employed modeling technique?
%     Authors are invited to suggest further requirements to demonstrate particular benefits of the approaches they adopt.
% \end{enumerate}
% }

% structure : plus ou moins imposée par le challenge (structure actuelle = sections au moins attendues) 