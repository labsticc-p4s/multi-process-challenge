Model-driven engineering (MDE) has traditionally adopted a two-level, top-down approach, where meta-models reside in a certain meta-level, and models are created one meta-level below by using types from the metamodel (e.g., EMF - \cite{emf} and MOF - \cite{omg2013mof}).

This strict approach shows its limitations when modeling complex domains which require more than one level of specialization, e.g., to adapt the model to application sub domains. Indeed, the two-level approach fails to acknowledge the existence of two different forms of classification (ontological vs linguistic) and the duality class-object, leading to the introduction of accidental complexity when dealing with the aforementioned complex domains.

In order to tackle this problem multi-level modeling have been introduced. Contrary to the strict approach, multi-level modeling advocates the use of a flexible number of levels as well as a flexiblzation of the instantiation concept.

This approach is gaining interest lately within the modeling community as shown by the MULTI level-modeling workshop and the daghtul seminar and number of tools. Taking advantage of this vibrant state of affairs, and in order to foster discussion and enable comparison between competing approaches, a multi-level modeling challenge has been created. This paper is a response to the latest multi-level modeling challenge which consist of modeling processes. We do so by using model federation and the OpenFlexo framework.

Model federation is a modeling approach in which...

Our solution, which is fully implemented and executable, meets all the requirements.

The rest of the paper is organized as follows.


% \cite{multiProcessChallenge2019}

% \emph{Submissions responding to the challenge should describe
% a multi-level model conforming to the case description,
% including justifications for non-trivial design decisions. In
% order to foster comparability between solutions, respondents
% are asked to make sure that concepts of the case description
% are explicitly represented by one or more model elements.}\\

% \emph{criteria:
% \begin{enumerate}
%     \item Does the response address the established domain as described in the challenge description and demonstrate the use of modeling features?
%     \item Does it evaluate/discuss its proposed modeling solution against the criteria described in the challenge description?
%     \item Does it discuss the merits and limitations of the employed modeling technique?
%     Authors are invited to suggest further requirements to demonstrate particular benefits of the approaches they adopt.
% \end{enumerate}
% }

% structure : plus ou moins imposée par le challenge (structure actuelle = sections au moins attendues) 