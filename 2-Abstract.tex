Model federation is a multi model \emph{management} approach based on the use of virtual models and loosely coupled links. The models in a federation remain autonomous and represented in their original technological spaces whereas virtual models and links serve as control components used to present different views to the users and  maintain synchronization. In this paper we tackle the \emph{process modeling challenge}, which consists in providing a solution to the problem of specifying and enacting processes. Solutions must fulfill a number of requirements for a process representation defined at an abstract process-definition level and at various more concrete domain-specific levels, resulting in a multi-level hierarchy of related models. We present a solution based on model federation and discuss the advantages and limitations of using this approach for multi-level modeling. Our solution is fully implemented with the Openflexo framework and its internal Flexo Modeling Language (FML). It fulfills all the challenge requirements.

\noteSM{J'ai fait un premier essai d'abstract... un peu trop vendeur peut-etre?}
\noteJC{bof, pas de manière éhontée : il faut qu'un abstract soit un peu
  vendeur et les exigences sont bien remplies. La fin est très positive, mais
afactuelle selon moi.}

