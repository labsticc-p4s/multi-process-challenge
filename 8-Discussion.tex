\noteJC{notes en vrac
  \begin{itemize}
    \item qui ? jc, fabien
    \item l'appel contraint plus ou monis la structure et le contenu de cette
      section
  \begin{itemize}
    \item discussing choices  made (ex : héritage vs attr), pointing out potential compromises/deficiencies ;
    \item \emph{Challenge respondents must discuss their multilevel
model solution with regard to the following aspects, each
of which should be treated in a specific sub-section of the
“Assessment” section of the article}
  \end{itemize}
    \item pas d'explicitation de niveaux de « concrétude » : conséquences ? facile de tracer des liens entre les niveaux ; difficile d'avoir le niveau ;
    \item outil adaptable
    \item niveau d'abstraction
    \item réutilisabilité des modèles ; 
    \item ajouter quelques métriques "gros grains"
    \item difficultés/facilités : contraintes faites à la main par rapport aux
      langages où on peut spécifier le niveau d'extension lorsque l'on trace un
      lien multi niveaux -> outil plus souple, mais moins de comportements
      automatisés
    \item + ouvert
    \item moins de support (faudrait implémenter ce qui manque)
    \item requiert de la méthode, ajouter des comportements, etc.
    \item point saillant :  outillage "gratuit" = on fournit une solution + l'outil pour exécuter le processus
\end{itemize}}
 
\todo[inline]{on pourrait aussi répondre ici aux 3 questions posées
explicitement (copiées dans l'intro pour le moment), a priori non, le chef
« trouve que c'est chiant »}

%
%%niveaux de ``concrétude'' -> level-blind solution
%
%Although our tool was not designed with a focus on multi-level modelling, we were able to propose a fully working solution. 
%
%Our approach is level-blind, therefore it is not easy to quickly retrieve the abstraction level of a concept.
%
%%métriques gros grains ?
%
%
%%choices, compromises/deficiences
%During the modeling phase, we had to choose 
%(for instance requirement xx 
%
%%+/-
%
%%points saillants


  %plan plus ou moins imposé par l'appel
\todo[inline]{mandatory discussion aspects (on verra plus tard si on fait des
sous-sections ou pas) ↓}

  \subsection{Basic modeling constructs}
  %Explain the basic modeling constructs used in the solution.

  \subsection{Levels}
  %Levels (or other model content organization schemes employed)

  %Explain the nature of “levels” in the model, how model elements are arranged
  %on these levels and which relationships (such as “instance-of”) may feature
  %between elements at different levels. The nature of levels should be
  %captured by explicitly stating the level segregation and the level cohesion
  %principles used [5]. Avoid vague language such as “higher level concepts are
  %more abstract” if the inter-level relationship is more specific. If the
  %inter-level relationship is deliberately allowed to be vague, state this
  %explicitly.

  The Openflexo approach is level-blind: ``levels'' have no specific
  nature and there are no numbered levels. 

  \subsection{Number of levels}
  %Elaborate whether the submitted solution could have had more or fewer levels
  %and explain how any potentially existing degrees of freedom were resolved.

  Due to the fact our approach is level-blind, our solution could have had more
  or fewer levels depending on the variations of the use case. %As levels are
  %not specifically encoded, 

  \subsection{Cross-level relashionships}

  %Discuss if and how associations and links can connect model elements at
  %different levels. State well-formedness constraints, if any apply.

  Thanks to the level-blindness nature of the Openflexo approach, cross-level
  relashionships are not an issue. Model elements of different levels can be
  linked each other in a transparent way. %If the user has defined
  %well-formedness constraints, then they are 

  \subsection{Cross-level constraints}

%Discuss if and how constraints can span multiple levels, especially with
%regard to cross-level relationships.

  As for cross-level relationships, cross-level constraints do not have a
  specific nature. They are like any constraint a user can define on a model
  element. Therefore, if a constraint is mandatory when establishing a
  relationship between elements from different levels, the user has to create
  it manually. This can be a limit of our approach: the flexibility of our
  tooling at the cost of a limited number of automated behaviors. 
  

  \subsection{Integrity mechanism}

  %Discuss how the integrity of level contents is preserved when changes to
  %level contents occur.

  In order to preserve the integrity of level contents when changes to level
  contents occur, a validation has to be triggered. The tool reports all
  inconsistencies that the user is free to fix. Note that in our approach, the
  metamodels are built in a ad hoc way  together with the models
  (co-construction), threfore the user validates on a ad hoc basis.

  \subsection{Deep characterization}

  %Discuss if and how higher levels influence elements at lower levels with a
  %level distance of two or more. Such an influence may be desired to ensure
  %properties of lower level elements regardless of the design choices that
  %modelers make at intermediate levels, including future extensions to
  %intermediate levels.

  \noteJC{pas trop clair pour moi pour le moment}

  \subsection{Generality}

  %Discuss the generality of the solution. Is (part of) it applicable to other
  %domains? Does it embody invariant principles of the domain(s) it covers with
  %minimal redundancy?

  Due to the nature of model federation, the models are highly reusable.
  Although we build ad hoc metamodels in our approach, our solution separates
  the domain-specific elements from general purpose ones, making easily
  possible to apply it to other domains. 

  \subsection{Extensibility}

  %Elaborate how the solution would respond to further requirements, such as
  %further special tasks that must be taken care of by special actors. Identify
  %expected extension points in the solution, e.g., subtyping opportunities. If
  %level insertion is a possibility in your chosen approach, explain how it
  %would be performed.

  A strength of the  model federation approach we have adopted resides in its
  flexibility. As we build metamodels and models together instead of fitting
  into a metamodel, our solution is more flexible. Therefore one can extend a
  solution easily. The challenge itself can be seen as a validation of the
  extensibility ability of our solution: it was decomposed into two steps
  (generic process Xsure, then ACME process) which is kind of simulation of the
  evolution of the  specification. We observed that our approach has been
  resilient to change.

  Due to the level-blind aspect of our approach, we could insert a new level,
  for instance. It would consist in creating new concepts we would link to
  other ones. Then, we would add any necessary constraints on those concepts
  and on the relashiontips linking them to the other existing concepts. 

  \noteJC{comment étendre : prendre l'exemple donné ?; level insertion possible
  : oui, vu que pas de notion de level}

  
\todo[inline]{recommended discussion aspects ↓}
\subsection{formalisms and tools ?}

%Indicate whether there are formalisms to establish the semantics of the MLM
%technique and/or tools that support the present solution

\noteJC{on a du tooling, un peu moins de formalisme}

Currently, there are no formalism to establish the semantics of the MLM
technique in our solution. \textcolor{red}{To develop…}

Our solution is fully tooled by a ``free'' ad hoc tool. We use the Openflexo
infrastructure to build models and metamodels together. In addition to the
solution, it also provides a dedicated tool which allows processes execution. 
\noteJC{dire qqch sur le fait que produire une solution avec son outil associé
  est une force, genre ``Providing a solution and its associated tool can be
…'' mais pour le moment je ne vois pas comment l'enchainer}

\subsection{Model verification}

%Discuss model verification (e.g., consistency analyses) or other quality
%assessment mechanisms supported by the MLM technique employed.

By default, our tooling does not include a lot of automated verifications.
However, a user can add verification mechanisms by writing new behaviors
attached to concepts. Presently, the verifications rely on the users when they
build the metamodels and the models.
