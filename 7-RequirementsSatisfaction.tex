%Satisfaction of Requirements: demonstration of how the solution satisfies the challenge requirements
Now we discuss how the solution satisfies the challenge requirements.

\noteSylvain{
Je voulais dire ici qu'on avait construit la solution pas à pas en suivant les requirements les uns après les autres, et qu'on avait à chaque fois veillé à traduire et respecter les requirements. Ca me semble très lourd de tout rejustifier ici.

On peut cependant expliciter les différentes techniques employées:
\begin{itemize}
    \item Requirements satisfaits par conformance syntaxique: obligation de respecter le modèle structurel
    \item Requirements satisfaits par ajout de contraintes: on peut instantier quelque chose de syntaxiquement correct, mais qui viole certaines contraintes (sur l'outillage les croix rouges)
    \item Requirements satisfaits par l'outillage: ajout de comportements de manipulation du modèle qui "forcent" à respecter les contraintes d'intégrité resultant des requirements.
\end{itemize}

Peut-être peut-on faire un tableau qui rescence comment on a satisfait chaque contrainte ? Est ce que c'est intéressant ?
}


\todo[inline]{OK pour le tableau synthétique +un petit peu de texte + une mini intro qui référence la section précédente (vu que les req ont été couverts dans la section précédente ; pas de fusion des deux sections}