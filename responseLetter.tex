% Title:    A LaTeX Template For Responses To a Referees' Reports
% Author:   Petr Zemek <s3rvac@gmail.com>
% Homepage: https://blog.petrzemek.net/2016/07/17/latex-template-for-responses-to-referees-reports/
% License:  CC BY 4.0 (https://creativecommons.org/licenses/by/4.0/)
\documentclass[10pt]{article}

% Allow Unicode input (alternatively, you can use XeLaTeX or LuaLaTeX)
\usepackage[utf8]{inputenc}
\usepackage{hyperref}
\usepackage{xcolor}
\usepackage{todonotes}

\usepackage{microtype,xparse,tcolorbox}
\newenvironment{reviewer-comment }{}{}
\tcbuselibrary{skins}
\tcolorboxenvironment{reviewer-comment }{empty,
  left = 1em, top = 1ex, bottom = 1ex,
  borderline west = {2pt} {0pt} {black!20},
}
\ExplSyntaxOn
\NewDocumentEnvironment {response} { +m O{black!20} } {
  \IfValueT {#1} {
    \begin{reviewer-comment~ }
      \setlength\parindent{2em}
      \noindent
      \ttfamily #1
    \end{reviewer-comment~ }
  }
  \par\noindent\ignorespaces\color{blue}
} { \bigskip\par }

\NewDocumentCommand \Reviewer { m } {
  \section*{Comments~by~Reviewer~#1}
}

\NewDocumentCommand \Editor { m } {
  \section*{Comments~by~Editor}
}

\NewDocumentCommand \ToolComments { m } {
  \section*{Tool~Comments~by~Reviewer~#1}
}

\ExplSyntaxOff
\AtBeginDocument{\maketitle\thispagestyle{empty}\noindent}

% You can get probably get rid of these definitions:
\newcommand\meta[1]{$\langle\hbox{#1}\rangle$}
\newcommand\PaperTitle[1]{``\textit{#1}''}

\title{Response letter to EMISAJ Manuscript entitled "Multi-level modeling with Openflexo/FML- A contribution to the MULTI process challenge"}
%\author{Author1 \and Author2 \and Author3}
\date{\today}

\begin{document}

Dear Editor,

\bigskip
Thanks for your message of August 16th, 2021, in which you suggested preparing a revision of our manuscript  \textbf{Multi-level modeling with Openflexo/FML- A contribution to the MULTI process challenge}. 

\bigskip
We are grateful to the reviewers for their interest, effort, and suggestions. We detail in the following pages how we have addressed the reviewers' concerns; we are confident that the paper is now improved, and hope that you will agree. We have organized the responses to the comments on a per-reviewer basis, focusing, as suggested, on: 

\begin{itemize}
\item Improving the presentation.
\item Provide a more explicit description of how the multi-level modelling emulation is achieved.
\item Discussing the general advantages and limitations of the multi-level emulation, including a clarification of what the implications would be of supporting more "levels" than the challenge required.
\end{itemize}

Moreover, the paper has been thoroughly revised to fix minor issues and to give the additional clarifications required by the reviewers. Please let us know if you need any further information or clarification.

\bigskip
Sincerely yours,

\bigskip
Sylvain, Guillaume, Caine, Joel, Jean-Christophe, Salvador, Fabien and Antoine

\pagebreak

\Editor{}

\begin{response}{the reviewers have made suggestions for improving the presentation but also raised a number of issues requiring clarification. An overarching theme is the request for a more explicit description of how the multi-level modelling emulation is achieved and what its general advantages and limitations are, including a clarification of what the implications would be of supporting more "levels" than the challenge required.} TODO.
\end{response}


\begin{response}{We very much welcome a "conventional" response to the challenge but would like to see the respective reviewer questions addressed in a revised version which is explicit as possible about the advantages and the disadvantages of the emulation. You may find the references listed below useful to keep explanations concise by referencing related work (note that reviewer 3’s “Type-Instance pattern” is also known as the “Type Object” pattern). You should also respond to other concerns reviewers raise, either by making changes to the submission or in the form of an authors' response.} TODO. We may have to create a new section.
\end{response}


\begin{response}{Finally, to achieve consistency with other submissions, could you please change the subtitle to "A Contribution to the Multi-Level Process Challenge"?} Of course we can.
\end{response}

\pagebreak

\Reviewer{\#1}
\begin{response}{In section 2 Technology the authors describe the foundations of the Openflexo approach. Here it would greatly help the reader to see an example how the approach works for traditional modeling languages. For example, how would you realize a small subset of BPMN with the approach? This could briefly be illustrated so as to give the reader an immediate understanding. Especially, the 'metamodel'-agnostic approach seems very interesting but I would like to see how that is used for traditional modeling approaches.}
TODO.
\end{response}

\begin{response}{Also, at the stage of section 2, the characterization of the FML language should be made clearer. The metamodel in Figure 2 seems mainly conceptual - but at the same time you refer to it as a basis for implementation. Could that be clarified? eg you have the class "Behavior" - which is a concept and I did not understand right away what it is composed of. On the other hand you have for example the class "ModelSlot" which seems like a very technical realization. Also, you note that roles have types but I could not find any types in the metamodel (?). Thus, I propose to include a more detailed version of your metamodel at this stage. Please note that EMISAJ does not have a page limitation, so no worries if that should require more space.} 
TODO.
\end{response}

\begin{response}{In Figure 5, the base metamodel for representing processes is shown - however it is not linked to the concepts you introduced before. Would it be possible to maybe graphically highlight how it relates to the metamodel in Figure 2? Besides, as multi-level modeling is much different to traditional modeling, you may want to consider using a different notation rather than UML class diagrams - as these are commonly known for traditional modeling approaches. When reading the paper, I actually marked the references to types in yellow in my copy - ie from ProcessType to TaskType and from ActorType to ActorType (on the attribute levels) - this helped me to see more clearly where you have references from attributes to types - but that's just my way of doing it.} 
TODO.
\end{response}

\begin{response}{Regarding Figure 6 and Figure 7: In Figure 7 you show the process incl. sequence connectors - however in Figure 6 these seem to be missing - maybe to reduce complexity? I would suggest to probably focus on a subset of the process from Figure 7 and also show how the connectors are realized. To me this would increase the comprehensibility / consistency. Ie I could then infer exactly, how elements from the concrete model have been realized in the background.} 
TODO.
\end{response}

\begin{response}{in Figure 9 you show the model editor. Could you please also add a screenshot of the metamodel editor?} 
TODO.
\end{response}

\begin{response}{There are some minor typos throughout the manuscript (eg also regarding the reference Jeusfeld2019; choose -> chose; axis -> axes; aN FML virtual model...; etc.)} 
TODO.
\end{response}


\pagebreak

\Reviewer{\#2}

\begin{response}{Here goes the question} 
TODO.
\end{response}

\pagebreak

\Reviewer{\#3}

\begin{response}{Here goes the question} And here goes the answer.
\end{response}






\end{document}

