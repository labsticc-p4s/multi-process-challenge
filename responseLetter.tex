% Title:    A LaTeX Template For Responses To a Referees' Reports
% Author:   Petr Zemek <s3rvac@gmail.com>
% Homepage: https://blog.petrzemek.net/2016/07/17/latex-template-for-responses-to-referees-reports/
% License:  CC BY 4.0 (https://creativecommons.org/licenses/by/4.0/)
\documentclass[10pt]{article}

% Allow Unicode input (alternatively, you can use XeLaTeX or LuaLaTeX)
\usepackage[utf8]{inputenc}
\usepackage{hyperref}
\usepackage{xcolor}
\usepackage{todonotes}

\usepackage{microtype,xparse,tcolorbox}
\newenvironment{reviewer-comment }{}{}
\tcbuselibrary{skins}
\tcolorboxenvironment{reviewer-comment }{empty,
  left = 1em, top = 1ex, bottom = 1ex,
  borderline west = {2pt} {0pt} {black!20},
}
\ExplSyntaxOn
\NewDocumentEnvironment {response} { +m O{black!20} } {
  \IfValueT {#1} {
    \begin{reviewer-comment~ }
      \setlength\parindent{2em}
      \noindent
      \ttfamily #1
    \end{reviewer-comment~ }
  }
  \par\noindent\ignorespaces\color{blue}
} { \bigskip\par }

\NewDocumentCommand \Reviewer { m } {
  \section*{Comments~by~Reviewer~#1}
}

\NewDocumentCommand \Editor { m } {
  \section*{Comments~by~Editor}
}

\NewDocumentCommand \ToolComments { m } {
  \section*{Tool~Comments~by~Reviewer~#1}
}

\ExplSyntaxOff
\AtBeginDocument{\maketitle\thispagestyle{empty}\noindent}

% You can get probably get rid of these definitions:
\newcommand\meta[1]{$\langle\hbox{#1}\rangle$}
\newcommand\PaperTitle[1]{``\textit{#1}''}

\title{Response letter to EMISAJ Manuscript entitled "Multi-level modeling with Openflexo/FML- A contribution to the MULTI process challenge"}
%\author{Author1 \and Author2 \and Author3}
\date{\today}

\begin{document}

Dear Editor,

\bigskip
Thanks for your message of August 16th, 2021, in which you suggested preparing a revision of our manuscript  \textbf{Multi-level modeling with Openflexo/FML- A contribution to the MULTI process challenge}. 

\bigskip
We are grateful to the reviewers for their interest, effort, and suggestions. We detail in the following pages how we have addressed the reviewers' concerns; we are confident that the paper is now improved, and hope that you will agree. We have organized the responses to the comments on a per-reviewer basis, focusing, as suggested, on: 

\begin{itemize}
\item Improving the presentation.
\item Provide a more explicit description of how the multi-level modelling emulation is achieved.
\item Discussing the general advantages and limitations of the multi-level emulation, including a clarification of what the implications would be of supporting more "levels" than the challenge required.
\end{itemize}

Moreover, the paper has been thoroughly revised to fix minor issues and to give the additional clarifications required by the reviewers. Please let us know if you need any further information or clarification.

\bigskip
Sincerely yours,

\bigskip
Sylvain, Guillaume, Caine, Joel, Jean-Christophe, Salvador, Fabien and Antoine

\pagebreak

\Editor{}

\begin{response}{the reviewers have made suggestions for improving the presentation but also raised a number of issues requiring clarification. An overarching theme is the request for a more explicit description of how the multi-level modelling emulation is achieved and what its general advantages and limitations are, including a clarification of what the implications would be of supporting more "levels" than the challenge required.} TODO.
\end{response}


\begin{response}{We very much welcome a "conventional" response to the challenge but would like to see the respective reviewer questions addressed in a revised version which is explicit as possible about the advantages and the disadvantages of the emulation. You may find the references listed below useful to keep explanations concise by referencing related work (note that reviewer 3’s “Type-Instance pattern” is also known as the “Type Object” pattern). You should also respond to other concerns reviewers raise, either by making changes to the submission or in the form of an authors' response.} TODO. We may have to create a new section.
\end{response}


\begin{response}{Finally, to achieve consistency with other submissions, could you please change the subtitle to "A Contribution to the Multi-Level Process Challenge"?} Of course we can.
\end{response}

\pagebreak

\Reviewer{\#1}
\begin{response}{Here goes the question} And here goes the answer.
\end{response}




\pagebreak

\Reviewer{\#2}

\begin{response}{Here goes the question} And here goes the answer.
\end{response}

\pagebreak

\Reviewer{\#3}

\begin{response}{Here goes the question} And here goes the answer.
\end{response}






\end{document}

