%Technology: precise description of the technology / approach that is used

% Inspiré de la section 10 du papier formose

\begin{figure}[t]
    \centering
    \includegraphics[width=\columnwidth]{Figures/federation.pdf}
    \caption{The model federation approach}
    \label{fig:mf}
\end{figure}

To meet the \mlpc, we have decided to use the language infrastructure of
model federation approach~\parencite{Golra2016-federation}. \emph{Model
  federation} is a way to assemble models using some kind of
low-coupling links. It has been studied to answer the \OMG' RFP
Semantic Modeling for Information Federation~\parencite{simf}. In
contrast to approaches that compose metamodels into a single large
metamodel grouping all needed entities, model federation build links
between models and metamodels (even through levels) to make ``things''
work together. As an illustration of this feature, we developed a
free-modeling editor -- freeing oneself from the bonds of
model/metamodel conformity -- that is presented
in~\parencite{models2016-freemodel}. Another notable feature of
this approach is the strong decoupling among tools that remain usable
after federations are made.
So to provide an adaptable framework, we have developed a flexible modeling language to provide modeling features without strict modeling levels. In a federation, the targeted modelling approaches can be unknown at the beginning and also can integrate several modeling levels.


We decided to use this approach since it offers the possibility to
link and to navigate among levels. Before we describe the overall
architecture in next section, here are some key concepts implemented
in the Openflexo~\parencite{openflexo_link} framework.

% figure
% Dans la suite les références à la figure sont commentés


\begin{figure*}
    \centering
    \includegraphics[width=1.0 \textwidth]{Figures/FMLMetaModel.pdf}
    \caption{Representation of the main concepts of {FML} language}
    \label{fig:mm}
\end{figure*}
%\noteAntoine{Dans la figure ne manque-t-il pas les containements ? (VM vers Concepts)}
%\noteSylvain{C'est le cas dans cette nouvelle figure}

This framework relies on the architecture of Figure~\ref{fig:mf}. A federation
gathers a set of conceptual models, named \emph{virtual models} and a
set of \emph{federated models}. Each federated model pertains to a
\emph{technological space} and uses the language of its specific
paradigm while a virtual model is built using the Flexo Modeling
Language (\FML). Each federated model is an autonomous
element that may evolve with its own tooling. The virtual models
serve as control elements binding the federated models together.
In this paper, we mainly use conceptual models and more precisely \FML its domain specific modeling language. The only use of the federation aspect is on the tooling made for the process language.

A UML like representation of a simplified metamodel of \FML is provided in Figure~\ref{fig:mm}. \FML is language designed to define models as virtual models. A \emph{virtual model} is composed of a set of \emph{concepts}, while itself being a \emph{concept}. Hence, \emph{virtual models} are structuring units forming architectures while \emph{concepts} are the core entities. A concept has a set of \emph{properties} and \emph{behaviors}. \emph{Properties} can be either simple variables, roles (pointers to a modeling element in a \emph{virtual model} or an external technology-specific model), or properties bound to complex control graphs. \FML is designed to define not only the structure of virtual models but also also the collection of actions that can be performed on them. These actions are called \emph{behaviors} and are composed with behavioural primitives called \emph{EditionAction} (core \FML primitives or behavioural primitives exposed by a specific technology). These \emph{behaviors} can either be called or triggered by events. The reactive behaviors are mainly useful when a federated model evolution needs to trigger a computation. Note that we do not exploit this possibility in the challenge.

A parallel to object-oriented approach can be useful to understand \FML\footnote{Some aspects of \FML do not exist in object oriented approach.}. A concept corresponds to a class, its properties to the attributes of the class and its behaviors to the methods of the class. These properties have types defining the kind of value the role will point at runtime. Whenever a type external to the federation space is used, one needs to use a \emph{model slot}. A model slot is a mediation entity in charge of giving access to external elements using a \emph{technology adapter}\footnote{It is a reusable library that defines connections between the \FML execution engine and a particular technological space.}.

% Sylvain : relire et vérifier

\noteSylvain{A la fin s'arranger pour que la figure 3 (fig:BPMNSubsetExample) et le listing soient cote a cote}

\begin{figure}[t]
    \centering
    \includegraphics[width=\columnwidth]{Figures/BPMNSubsetExample.pdf}
    \caption{An example of BPMN subset showing both the model and an instance of that model}
    \label{fig:BPMNSubsetExample}
\end{figure}

Figure \ref{fig:BPMNSubsetExample} shows how to realize a small subset of BPMN with Openflexo/FML. Concretely, we have designed a \textit{virtual model} (SimplifiedBPMN) with 3 \textit{concepts} representing a process and its contained tasks. An instance of this \textit{virtual model} is presented in the bottom of the figure, showing instances with their instanceOf relationships with their model.

Following listing shows the textual representation of the above described SimplifiedBPMN \textit{virtual model}.

% Sylvain : relire et vérifier


\begin{lstlisting}
// A simplified BPMN metamodel exposing
// both concepts BPMNProcess and BPMNTask
model SimplifiedBPMN {

  // Abstract concept BPMNElement defining a name
  abstract concept BPMNElement {
    String name;
    // A basic constructor
    create(String name) {
      this.name = name;
    }
  }

  concept BPMNProcess extends BPMNElement {

    // A basic constructor
    create(String name) {
      super(name);
    }
    // Describe here other properties and
    // behaviours for BPMNProcess concept
    ...

    concept BPMNTask extends BPMNElement {

      // A basic constructor
      create(String name) {
        super(name);
      }
      // Encodes execution of task
      execute() {
        log("Execute task "+name);
        ...
      }
      // Describe here other properties and
      // behaviours for BPMNTask concept
      ...

    }

    // Other core concepts
  }
}
\end{lstlisting}


When the \FML execution engine runs a federation, it creates virtual
model instances containing concept instances. Some concept instances
are connected to external elements through model slot instances.


The tool support for model federation framework,
Openflexo\footnote{\url{https://github.com/openflexo-team}}, is
developed as an open source initiative. This tool offers a \FML
execution engine with an interactive virtual model design environment.

Finally, tools have their own model. We have taken advantage of
Openflexo features to build, in parallel with the models  required by the challenge, a drawing tool that makes our
solution (partially) executable.
