%Satisfaction of Requirements: demonstration of how the solution satisfies the challenge requirements

In the previous Sect.~\ref{sec:model}, we demonstrated that our solution
satisfies all the challenge requirements. In this section, we summarize the
different techniques used to satisfy them, without repeating all the
justifications. We classify these techniques into three categories:
\begin{itemize}
    \item \emph{syntactic conformance}: the requirements are structurally
      satisfied when the model conforms to its metamodel,
    \item \emph{constraints checking}: the requirements are fulfilled when all
      instances satisfy both syntactic conformance and the evaluation of
      invariants expressed for related concepts,
    \item \emph{tooling}: the requirements are enforced by the implementation
      of the associated tool.
\end{itemize}


Tab.~\ref{tab:RequirementsSatisfactionBaseMetamodel} summarizes the requirement
coverage for the base metamodel illustrated by XSure insurance use case while
Tab.~\ref{tab:RequirementsSatisfactionAcme} shows requirements satisfaction for
Acme software engineering process. Syntactic conformance is more precisely
classified into six subcategories:

\begin{itemize}
    \item \emph{Conceptualization} (\textit{Conc.}): a concept is created to
      fulfill or meet the requirement (for example for \reqp{1}:
      \textit{process type} is conceptualized in a concept
    \texttt{ProcessType}).

    \item \emph{Specialization} (\textit{Spec.}): a concept is specialized in
      the sense of object-oriented modeling (for example for \reqp{2}:
      \texttt{Gateway} defines an abstract behavior \texttt{execute()} and
    \texttt{Sequencing}, \texttt{AndSplit}, \texttt{AndJoin}, \texttt{OrSplit}
  and \texttt{OrJoin} are defined as inherited concepts implementing specific
business logic).

    \item \emph{Composition} (\textit{Comp.}): some concepts are in relation,
      or specific attributes were added to some concepts (for example for
      \reqp{3}: a \textit{process type} has one \textit{initial task type}).

    \item \emph{Linguistic instantiation} (\textit{L.Inst.}): a requirement is
      satisfied by the instantiation of one or more concepts (for example for
      \reqs{1}: \inst{Requirement analysis} is defined as an instance of
    \texttt{SETaskType}).

    \item \emph{Ontological instantiation} (\textit{O.Inst.}): fulfillment of a
      requirement is obtained by the relation between an instance and its type
      definition instance (for example the notion of \textit{developer} of
      \reqs{3} is implemented by both a concept in \textit{Acme metamodel}
    and an instance in \textit{Acme model}, see bottom left of
  Fig.~\ref{fig:AcmeFullArchitecture}).

    \item \emph{Behavioral modeling} (\textit{B.Mod.}): a requirement is
  satisfied by the definition of one or more behaviors, which may be combined
  with constraints checking and associated tools (for example \reqp{17}).
  \end{itemize}

%\noteSylvain{updated}

%Tab.~\ref{tab:RequirementsSatisfactionBaseMetamodel}
%Tab.~\ref{tab:RequirementsSatisfactionAcme}

%Now we discuss how the solution satisfies the challenge requirements.

%\noteSylvain{
%Je voulais dire ici qu'on avait construit la solution pas à pas en suivant les requirements les uns après les autres, et qu'on avait à chaque fois veillé à traduire et respecter les requirements. Ca me semble très lourd de tout rejustifier ici.

%On peut cependant expliciter les différentes techniques employées:
%\begin{itemize}
%    \item Requirements satisfaits par conformance syntaxique: obligation de respecter le modèle structurel
%    \item Requirements satisfaits par ajout de contraintes: on peut instantier quelque chose de syntaxiquement correct, mais qui viole certaines contraintes (sur l'outillage les croix rouges)
%    \item Requirements satisfaits par l'outillage: ajout de comportements de manipulation du modèle qui "forcent" à respecter les contraintes d'intégrité resultant des requirements.
%\end{itemize}

%Peut-être peut-on faire un tableau qui rescence comment on a satisfait chaque contrainte ? Est ce que c'est %intéressant ?
%}
%\todo[inline]{OK pour le tableau synthétique +un petit peu de texte + une mini intro qui référence la section précédente (vu que les req ont été couverts dans la section précédente ; pas de fusion des deux sections}

%CT : Conceptualization
%SP : Specialization
%CP : Composition
%LI : Linguistic instantiation
%OI : Ontologic instantiation
%BM : behavioral modeling

\renewcommand{\arraystretch}{1.1}
\begin{table*}
 \centering
 \rowcolors{4}{tableShade}{white}  %% start alternating shades from 3rd row
\begin{tabular}{|c|c|c|c|c|c|c|c|c|}
   \hline
    & \multicolumn{6}{c|}{Syntactic conformance} & Constraints & \multirow{2}{*}{Tooling}\\
   \cline{1-7}
                 & Conc.      & Spec.      & Comp.      & L.Inst.     & O.Inst.     & B.Mod.   &  checking & \\
  \hline
    \reqp{1}  & \checkmark &            & \checkmark &            &            &            &            & \\
    \reqp{2}  & \checkmark & \checkmark & \checkmark &            &            &            &            & \\
    \reqp{3}  &            &            & \checkmark &            &            &            &            & \\
    \reqp{4}  & \checkmark &            &            &            & \checkmark &            &            & \\
    \reqp{5}  & \checkmark &            & \checkmark &            &            &            &            & \\
    \reqp{6}  & \checkmark &            & \checkmark &            &  &            &            & \\
    \reqp{7}  & \checkmark &            & \checkmark &            &            &            &            & \\
    \reqp{8}  &            &            & \checkmark &            &            &            &            & \\
    \reqp{9}  &            &            & \checkmark &            &            &            & \checkmark & \\
    \reqp{10} & \checkmark &            &            &            & \checkmark &            &            & \\
    \reqp{11} & \checkmark &            & \checkmark &            & \checkmark &            &            & \\
    \reqp{12} &            &            & \checkmark &            &            &            &            & \\
    \reqp{13} & \checkmark &            & \checkmark &            & \checkmark &            &            & \\
    \reqp{14} & \checkmark &            &            &            & \checkmark &            &            & \\
    \reqp{15} & \checkmark &            &            &            & \checkmark &            &            & \\
    \reqp{16} & \checkmark &            &            &            & \checkmark &            &            & \\
    \reqp{17} & \checkmark           &            &            &            &            & \checkmark & \checkmark & \checkmark \\
    \reqp{18} & \checkmark           &            &            &            &            & \checkmark &            & \\
    \reqp{19} &            & \checkmark & \checkmark &            &            & \checkmark &            & \\
  \hline
\end{tabular}
     \caption{Requirements satisfaction for base metamodel}
    \label{tab:RequirementsSatisfactionBaseMetamodel}
\end{table*}


\begin{table*}
 \centering
  \rowcolors{4}{tableShade}{white}  %% start alternating shades from 3rd row
\begin{tabular}{|c|c|c|c|c|c|c|c|c|}
   \hline
    & \multicolumn{6}{c|}{Syntactic conformance} & Constraints & \multirow{2}{*}{Tooling}\\
   \cline{1-7}
                 & Conc.      & Spec.      & Comp.      & L.Inst     & O.Inst     & B.Mod      & checking   & \\
  \hline
    \reqs{1}  &            &            &            & \checkmark &            &            &            & \\
    \reqs{2}  &            &            &            & \checkmark &            &            &            & \\
    \reqs{3}  & \checkmark & \checkmark & \checkmark & \checkmark &            &            &            & \checkmark \\
    \reqs{4}  & \checkmark & \checkmark & \checkmark & \checkmark &            &            &            & \\
    \reqs{5}  &            & \checkmark &            & \checkmark &            & \checkmark &            & \\
    \reqs{6}  &            & \checkmark &            & \checkmark &            & \checkmark & \checkmark & \\
    \reqs{7}  &            &            &            & \checkmark &            & \checkmark &            & \\
    \reqs{8}  &            &            &            & \checkmark &            &            &            & \\
    \reqs{9}  &            &            &            &            &            & \checkmark & \checkmark & \\
    \reqs{10} & \checkmark & \checkmark & \checkmark &            &            &            &            & \\
    \reqs{11} &            &            &            & \checkmark &            &            &            & \checkmark \\
    \reqs{12} &            &            &            & \checkmark &            &            &            & \checkmark \\
    \reqs{13} &            &            &            & \checkmark &            &            &            & \\
  \hline
\end{tabular}
     \caption{Requirements satisfaction for Acme software engineering process}
    \label{tab:RequirementsSatisfactionAcme}
\end{table*}
