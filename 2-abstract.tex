Model federation is a multi model \emph{management} approach based on the use of virtual models and loosely coupled links. The models in a federation remain autonomous and represented in their original technological spaces whereas virtual models and links (which are not level bounded) serve as control components used to present different views to the users and  maintain synchronization. In this paper we tackle the \emph{MULTI process modeling challenge}, which consists in providing a solution to the problem of specifying and enacting processes. Solutions must fulfill a number of requirements for a process representation defined at an abstract process-definition level and at various more concrete domain-specific levels, resulting in a multi-level hierarchy of related models. We present a solution based on model federation and discuss the advantages and limitations of using this approach for multi-level modeling. Concretely, we use virtual models and more precisely the Flexo Modeling Language (\FML) that serves to describe them as the main building block in order to solve the process modeling challenge whereas the federation aspect is used as a means to provide editing tooling for the resulting process language. Our solution is fully implemented with the Openflexo framework. It fulfills all the challenge requirements.

%\noteSylvain{Ce qui m'embête un peu avec cet abstract c'est qu'on met en avant la fédération de modèles alors que sur ce cas précis on en fait pas vraiment (sauf pour la partie outillage avec les éditeurs diagrammatiques). Concernant la fédération de modèle proprement dite, je crois qu'il faut qu'on montre qu'on a besoin d'une couche conceptuelle (qu'on relie à l'espace technique via les connecteurs techniques, les model slots et les rôles). C'est ce besoin qu'adresse le language \FML. On montre ici plutôt que le language \FML - utilisé comme language conceptuel - est assez intéressant, assez expressif pour capturer les exigences du challenge d'une part, et va jusqu'à permettre l'exécution des processus. Et cerise sur le gâteau: on peut construire un outillage graphique pour editer et exécuter ce use case.}
%\noteSylvain{En revanche, on peut arguer que pouvoir au même niveau manipuler modèle et métamodèle nous simplifie grandement les choses pour adresser des problématiques multi-niveaux > un bout de la conclusion}.
%The models in a federation remain autonomous and represented in their original technological spaces %whereas virtual models and links serve as control components used to present different views to the %users and  maintain synchronization. 
