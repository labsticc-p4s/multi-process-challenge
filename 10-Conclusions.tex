%Conclusions: including  lessons  learned,  impulses  for future work, etc.

%\todo[inline]{lessons learned, impulse for future work -> fait émerger la question d'ajouter des bibliothèques dédiées au multi-niveaux (avoir des behaviors)}

We fulfill all the challenge requirements and we propose a tooling that demonstrates the usability of our solution.  We developed an ad-hoc solution, including models and metamodels, thanks to the flexibility provided by the metametamodeling infrastructure offered by Openflexo.

As a future work we envision to explore a number of alternative solutions: first, we intend to propose another solution based on a more level-agnostic approach such as Clabjects. Instead of two levels defined by Type and Instance concepts, we could have a single concept mixing a type part and an instance part. This approach would probably ease adaptation if the problem specifications evolve. Secondly, another possible approach is the use of the free modeling tool proposed by Openflexo. The solution would be developed from examples (Acme and XSure processes for instance) from which models would be identified. Finally, the realisation of this challenge highlighted the interest of integrating specific behaviors for the management of multi-level concepts in the Openflexo infrastructure (as opposed to tailoring them as we did in this exercise). We intend to do so.