%Related Work: Positioning  and  contrasting  the  presented solution with related work

%Salvador commence
A plethora of multi-level modeling approaches and tools with different foundations have appeared in the recent years~\footnote{\url{https://homepages.ecs.vuw.ac.nz/Groups/MultiLevelModeling/MultiTools}}. Comparing them lies out of the scope of this paper. In this sense, we limit the following discussion to the presentation and comparison of previous solutions to the MULTI Process Challenge.

A first description of process modeling as a multi-level modeling problem was proposed by \cite{lara2018refactoring} in the context of a catalogue of refactoring for multi-level models (in a simpler form with less constraints and requirements w.r.t. the challenge version). A solution is provided with MetaDepth~\citep{metadepth}, which supports modeling with any number of levels, dual ontological/linguistic typing and deep characterization through potency.

\cite{multiecore2019} use MultiEcore \citep{multecore2016} to provide a solution to the challenge. MultEcore uses an extension of the two-level cascading technique and potency. They do not need the synthetic typing relations we introduce. However they need to do frequent model transformations to transform instance models into instantiable models. With respect to the proposed solution, we use less models as we do not include a generic software engineering process model nor a generic assurance process model. %Finally, they use multi-level couple model transformations for the definition and enforcement of constraints.

More similar to us, \cite{deeptelos2019} use DeepTelos \citep{deeptelos2016}, an extension of the Telos language \citep{telos1990} in order to solve the process challenge. As with Openflexo and \FML, they do not use explicit (numbered) levels nor potency. However, unlike Openflexo, DeepTelos integrates a multi-level modeling specific construct similar to the PowerType \citep{atkinson2001essence} pattern they call \emph{most general instances}.

%Their solution to the challenge includes 5 (ontological) levels and 8 models, including Ecore at the top. %They include two supplementary hierarchies (linguistic extensions) that are not bound to any of the %aforementioned 5 levels. Finally, they use multi-level couple model transformations for the definition and %enforcement of constraints. Their approach is implemented on top of EMF and achieves integration with its %strict modeling paradigm by extending the two-level cascading mechanism to all the hierarchy.

Similarly, \cite{dmla2019} use the Dynamic Multi-Layer Algebra (DMLA) \citep{dmla2017} in order to solve the process challenge. DMLA is a level-blind modeling framework which is fully customizable (\eg different types of instantiaton may be implemented). Their proposed solution separates the process challenge in two separate domains, namely, the tasks definition domain and the process definition domain. Wrappers are used in order to reuse tasks in processes. Their solution does not use inheritance, as it is not supported by the framework. It does not use explicitly separated models either.

%DMLA does not directly supports inheritance nor bi-directional navigation which prevents an easy solution to %some of the challenge requirements (e.g., p9 and s13).

%\noteSM{Say something about constraints in \FML} Bah... everybody does multilevel constraints.
