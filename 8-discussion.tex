% Assessment of the Modeling Solution: discussing choices made, pointing out potential compromises / deficiencies

%\noteJC{notes en vrac
%  \begin{itemize}
%    \item qui ? jc, fabien
%    \item l'appel contraint plus ou moins la structure et le contenu de cette
%      section
%  \begin{itemize}
%    \item discussing choices  made (ex : héritage vs attr), pointing out potential compromises/deficiencies ;
%    \item \emph{Challenge respondents must discuss their multilevel
%model solution with regard to the following aspects, each
%of which should be treated in a specific sub-section of the
%“Assessment” section of the article}
%  \end{itemize}
%    \item[\checkmark] pas d'explicitation de niveaux de « concrétude » : conséquences ? facile de tracer des liens entre les niveaux ; difficile d'avoir le niveau ;
%    \item outil adaptable
%    \item niveau d'abstraction
%    \item[\checkmark] réutilisabilité des modèles ;
%    \item ajouter quelques métriques "gros grains"
%    \item[\checkmark] difficultés/facilités : contraintes faites à la main par rapport aux
%      langages où on peut spécifier le niveau d'extension lorsque l'on trace un
%      lien multi niveaux -> outil plus souple, mais moins de comportements
%      automatisés
%    \item + ouvert
%    \item[\checkmark] moins de support (faudrait implémenter ce qui manque)
%    \item requiert de la méthode, ajouter des comportements, etc.
%    \item[\checkmark] point saillant :  outillage "gratuit" = on fournit une solution + l'outil pour exécuter le processus
%\end{itemize}}

%\todo[inline]{on pourrait aussi répondre ici aux 3 questions posées
%explicitement (copiées dans l'intro pour le moment), a priori non, le chef
%« trouve que c'est chiant »}

%\noteJC{pour le moment, seulement réponse aux différents points, on ne voit pas
%de discussion sur les choix, sur les potentiels compromis/déficiencies}

In this section, we discuss our multilevel model solution with regard to the required aspects
mentioned in the challenge.

  %plan plus ou moins imposé par l'appel
%\todo[inline]{mandatory discussion aspects}

  \subsection{Basic modeling constructs}
  %Explain the basic modeling constructs used in the solution.

%  \noteJC{pas clair pour moi ce qui est entendu par « basic modeling constructs », ce qui est dans la partie 2 ?}

  Our solution uses the basic modeling constructs depicted by the \FML core
  metamodel (Figure~\ref{fig:mm}) and described in
  the section~\ref{sec:technology}. Models are \emph{virtual models}. They are
  composed of \emph{concepts}, being themselves concepts. Concepts may have
  \emph{roles} and \emph{behaviors} (actions one can perform).

  In order to provide the graphical tools, our solution also uses \emph{model
  slots} to connect some concept instances to external elements (in our case:
  instances of diagram from our diagraming TA). The use of slots has been
  described in the section~\ref{sec:model} and is illustrated by bold circles in
  the Figure~\ref{fig:ToolingArchitecture}.


  \subsection{Levels}
  %Levels (or other model content organization schemes employed)

  %Explain the nature of “levels” in the model, how model elements are arranged
  %on these levels and which relationships (such as “instance-of”) may feature
  %between elements at different levels. The nature of levels should be
  %captured by explicitly stating the level segregation and the level cohesion
  %principles used [5]. Avoid vague language such as “higher level concepts are
  %more abstract” if the inter-level relationship is more specific. If the
  %inter-level relationship is deliberately allowed to be vague, state this
  %explicitly.

  The Openflexo approach is level-agnostic: ``levels'' have no specific nature
  and there are no numbered levels. In our solution, concepts of a given level
  are grouped into a virtual model. Inheritance and instantiation allow the
  establishment of relationships between concepts from different levels.
%\ref{fig:BPMNSubsetExample}
%  Figure~\ref{fig:AcmeFullArchitecture} shows three levels: the metamodel, the
%  ACME metamodel and the ACME model.
Figure~\ref{fig:MultilevelArchitecture} shows the multilevel architecture of
our solution.
\noteJC{updated}

%\todo[inline]{mais on peut dire que les niveaux sont empaquetés dans des virtual models ; pour passer d'un niveau à l'autre, on a des relations : héritage et instantiation}

  \subsection{Number of levels}
  %Elaborate whether the submitted solution could have had more or fewer levels
  %and explain how any potentially existing degrees of freedom were resolved.

  Due to the fact that our approach is level-agnostic, our solution could have
  more or fewer levels depending on the variations of the use case. However,
  the number of levels is related to the problem, making the approach fixed (to
  3).

  \subsection{Cross-level relationships}

  %Discuss if and how associations and links can connect model elements at
  %different levels. State well-formedness constraints, if any apply.

  Thanks to the level-agnosticism nature of the Openflexo approach, cross-level
  relationships are not an issue. Model elements of different levels can be
  linked each other in a transparent way, using inheritance or instantiation.
  For instance in the figure~\ref{fig:AcmeFullArchitecture}, the
  \texttt{Developer} concept from \texttt{Acme} metamodel specializes the
  \texttt{ActorType} of the base metamodel, and the \texttt{Developer} of the
  \texttt{Acme} model is an instance of the metamodel concept.
%  \textcolor{red}{Phrase faite a relire sur Developer\dots}
\noteJC{updated}

  \subsection{Cross-level constraints}

%Discuss if and how constraints can span multiple levels, especially with
%regard to cross-level relationships.

  As for cross-level relationships, cross-level constraints do not have a
  specific nature. They are like any constraint a user can define by adding a
  behavior to a model element. Therefore, if a constraint is mandatory when
  establishing a relationship between elements from different levels, the user
  has to create it manually. This can be a limit of our approach: the cost of
  the flexibility of our tooling is a limited number of automated behaviors.

  %the flexibility of our tooling at the cost of a limited number of automated behaviors.


  \subsection{Integrity mechanism}

  %Discuss how the integrity of level contents is preserved when changes to
  %level contents occur.

  Behaviors are continuously checked, ensuring the integrity of the models when
  changes occur. However, most of these behaviors have to be written by the
  users. Thus, the integrity of the contents relies on the users when they
  build the metamodels and the models. The section~\ref{sec:model} describes
  how the requirements are captured and how behaviors are implemented with
  \FML: for instance, \texttt{isAuthorizedActor} behavior is needed to enforce
  \textbf{P17} requirement, leading to writing \FML code in \texttt{TaskType}
  and \texttt{Task}. Note that in our approach, the metamodels are built in an
  ad hoc way together with the models (co-construction), therefore the user
  also validates on an ad hoc basis.
\noteJC{updated}  

  \subsection{Deep characterization}

  %Discuss if and how higher levels influence elements at lower levels with a
  %level distance of two or more. Such an influence may be desired to ensure
  %properties of lower level elements regardless of the design choices that
  %modelers make at intermediate levels, including future extensions to
  %intermediate levels.

  %\noteJC{pas trop clair pour moi pour le moment -> N/A ? }

  Due to the nature of our approach and its level-agnosticism, \emph{deep
  characterization} does not apply to our solution. Such a mechanism could
  probably be encoded by specific concepts and constraints (behaviors).
  However, it would not be a generic mechanism, making it difficult to reuse
  for another problem.  Plus, programming the notion of \emph{depth} would
  probably imply that a virtual model knows its sub-elements, which is not
  desired and considered as bad design.
\noteJC{updated}

  \subsection{Generality}

  %Discuss the generality of the solution. Is (part of) it applicable to other
  %domains? Does it embody invariant principles of the domain(s) it covers with
  %minimal redundancy?

  Due to the nature of model federation, the models are highly reusable.
  Although we build ad hoc metamodels in our approach, our solution separates
  the domain-specific elements from general purpose ones, making possible to
  reuse the general concepts to other domains.

  \subsection{Extensibility}

  %Elaborate how the solution would respond to further requirements, such as
  %further special tasks that must be taken care of by special actors. Identify
  %expected extension points in the solution, e.g., subtyping opportunities. If
  %level insertion is a possibility in your chosen approach, explain how it
  %would be performed.

  A strength of the  model federation approach we have adopted resides in its
  flexibility. As we build metamodels and models together instead of fitting
  into a metamodel, our solution is more flexible. Therefore one can extend a
  solution easily. The challenge itself can be seen as a validation of the
  extensibility ability of our solution: it was decomposed into two steps
  (Xsure process, then ACME process) which can be seen as a simulation of the
  evolution of the  specification. We observed that our approach has been
  resilient to change.

  Due to the level-agnostic aspect of our approach, we could insert a new
  level, for example. It would consist in creating new concepts we would link
  to other ones. Then, we would add any necessary constraints on those concepts
  and on the relationships linking them to the other existing concepts. This
  type of change can be done without much effort. On the other hand, making
  changes to the existing models could be difficult. For example, given a
  requirement, an instance cannot be changed into a concept easily. This case
  could occur if the challenge had refined the specification by requiring
  specific design tasks.  In this context, we would have to transform the
  \inst{Design Task} instance into a concept whose instances could have been
  "using agile methodology".

  %\noteJC{comment étendre : prendre l'exemple donné ?; level insertion possible
  %: oui, vu que pas de notion de level}


%\todo[inline]{recommended discussion aspects}
\subsection{Tools}

%Indicate whether there are formalisms to establish the semantics of the MLM
%technique and/or tools that support the present solution

We use the Openflexo infrastructure to build models and metamodels together,
and to provide dedicated tooling to edit a Process type, to enact a process and
to execute it. Being able to provide a solution and its associated tools
quickly and easily is a noticeable feature.  We could also have taken advantage
of the model federation to connect our solution to other tools dedicated to
process edition and to process execution (\eg a \BPMN engine). However we
already had all the necessary components to provide the aforementioned
graphical tools.

\subsection{Model verification}

%Discuss model verification (e.g., consistency analyses) or other quality
%assessment mechanisms supported by the MLM technique employed.

Our tooling includes mechanisms to verify the models. First, syntactical
consistency is realized by cardinality checks and by typing. Second, the
constraints the users have written are continuously checked during model
edition and enactment. Thus, a part of the verification relies on the fact that
users add behaviors when they build the metamodels and the models. For
instance, \textbf{S6} requirement is constrained by an invariant implemented in
\texttt{CodeArtifact}. It ensures that \texttt{artifactType} has the right type
and that it implements the \texttt{languages} relation.
\noteJC{updated}
