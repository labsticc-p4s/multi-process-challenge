%Conclusions: including  lessons  learned,  impulses  for future work, etc.

%\todo[inline]{lessons learned, impulse for future work -> fait émerger la question d'ajouter des bibliothèques dédiées au multi-niveaux (avoir des behaviors)}

We fulfill all the challenge requirements and we propose a tooling that
demonstrates the usability of our solution. We developed an ad hoc solution,
including models and metamodels, thanks to the flexibility provided by the
metametamodeling infrastructure offered by Openflexo. This infrastructure helped us to overcome the limitations of the strict modeling paradigm mentioned in Section \ref{sec:introduction}, this is, the lack of support for different forms of classifications and for the duality type/object. In that sense, our solution seamlessly integrates the type-object pattern which allows us to dynamically create different \emph{instances} of model elements (e.g., \emph{TaskType}, among others) and use them for typing other model elements. This, together with the FML core support for (level agnostic) classical linguistic instantiation, permitted us to merge linguistic and ontological instantiation in order to provide a satisfactory solution to the Multi-Level Process challenge.

\textcolor{red}{The features of \FML enable a great modelling flexibility which, coupled with an appropriate methodology, allow us to achieve
the multi/level capabilities without dedicated tooling in order to solve problems which require it. We use virtual models to "implement" levels on demand.}
%\noteSM{Say that the flexibility of our approach coupled with a methodology allows us to achieve multi/level capabilities without specific tooling in order to solve problems which require it.}

As a future work we envision to explore a number of alternative solutions:
first, we intend to propose another solution based on a more level-agnostic
approach such as Clabjects. Instead of two levels defined by Type and Instance
concepts, we could have a single concept mixing a type part and an instance
part. This approach would probably ease adaptation if the problem
specifications evolve. Secondly, another possible approach is the use of the
free modeling tool proposed by Openflexo. The solution would be developed from
examples (Acme and XSure processes for instance) from which models would be
identified. 

Finally, the realization of this challenge highlighted the interest
of integrating specific behaviors for the management of multi-level concepts in
the Openflexo infrastructure. %We intend to do so.
