% detailed  presentation of a model, including justifications for design decisions

Figure \ref{fig:MultilevelArchitecture} shows the overview of the multivel architecture designed to capture the two use cases described in the Process Challenge.

% Expliquer que le use case XSure ne nécessitait pas l'extension de metamodèle requise pour le usecase Acme Software Developement Process

\subsection{Base metamodel for the Process Challenge}

\begin{figure*}
 \centering
    % \includegraphics[width=1.0 \columnwidth]{Figures/Metamodel.pdf}
    \includegraphics[width=1.0 \textwidth]{Figures/Metamodel.pdf}
     \caption{Process management base metamodel}
    \label{fig:BaseMetamodel}
\end{figure*}

\todo{Augmenter la taille des fontes}

In this section, we present and illustrate the base metamodel presented in figure \ref{fig:BaseMetamodel} with \textit{XSure} insurance domain use case, whose a partial description was provided in the challenge description. Note that the capture of the requirements \textbf{P1} to \textbf{P19} in the context of XSure insurance domain are straightforward implementable while instantiating \textit{XSure model}, as an instance of base metamodel (the left side of figure \ref{fig:MultilevelArchitecture}).

Figure \ref{fig:BaseMetamodel} represents this base metamodel with a UML-like formalism, well adapted to represent object-oriented FML concepts and their instances. Base conceptual element is called a \textit{FlexoConcept} in FML language, and is similar to the concept of class in object-oriented modeling. We use the UML representation of class to represent a \textit{FlexoConcept}, while cardinalities are shown using the UML-like classical rules.

Our proposition relies on ontologic instantiation as presented in figure \ref{fig:LinguisticAndOntologicInstantiation}, with a common root concept \textit{ModelingElement}. Two types of ontological instantiation are needed to meet the requirements of the challenge. One provides that some instances conform to only one type (\textit{Process} and \textit{Task} do respectively conform to one \textit{ProcessType} and \textit{TaskType}) while some entities define their conformity to several types (\textit{Actor} and \textit{Artifact} do respectively conform to several \textit{ActorType} and \textit{ArtifactType}). This instantiation is reified with the relations \textit{type} and \textit{types} between \textit{Type}, \textit{Instance} and \textit{MultiInstance} concepts.  

\subsubsection{Process type definition}

We first present the process definition part of base metamodel, located left of figure \ref{fig:BaseMetamodel}.

% Illustrer avec des instances de XSure

\textit{ProcessType} is defined as a specialization of Type \textit{FlexoConcept}, and references a collection of \textit{TaskType} through the composition relation \textit{taskTypes} with (0..*) cardinality (\textbf{P1}). A \textit{TaskType} is embedded in a \textit{ProcessType} and inherits from it context. To illustrate this in \textit{XSure} insurance domain use case, \textit{XSure model} defines \textit{Claim Handling}, instance of \textit{ProcessType}, and \textit{Receive Claim}, \textit{Asses Claim} and \textit{Pay premium}, instances of \textit{TaskType}.

\textit{ProcessType} also references a collection of gateways, reified with the \textit{Gateway} concept hierarchy. \textit{Gateway} is defined as a specialization of \textit{Type} and is specialized with \textit{Sequencing}, \textit{AndSplit}, \textit{AndJoin}, \textit{OrSplit} and \textit{OrJoin} concepts (\textbf{P2}). Depending of its type, a \textit{Gateway} defines one or more inputs and one or more outputs, depending of its underlying execution semantics. A \textit{ProcessType} additionally exposes unique initial \textit{TaskType} and a collection of final \textit{TaskType} with both attributes \textit{initialTaskType} (single cardinality) and \textit{finalTaskType} (cardinality 0..*) (\textbf{P3}). TaskType exposes a creator attribute as an association relation to Actor concept (\textbf{P4}), which is in the same conceptual level. 

\textit{ActorType} is defined as a sub-concept of \textit{Type} and the \textit{allowedActorTypes} relation to \textit{ActorType} which is defined in \textit{TaskType} (with 0..* cardinality) captures \textbf{P5} requirement. Requirement \textbf{P6} is symmetrically satisfied with \textit{allowedActors} relation to \textit{Actor} also defined in \textit{TaskType} (with same 0..* cardinality). Same modeling pattern applies for \textit{ArtefactType} defined as a sub-concept of \textit{Type}, and both relations \textit{usedArtifactTypes} and \textit{producedArtifactTypes} defined in \textit{TaskType} (\textbf{P7}). \textit{TaskType} additionnaly exposes an \textit{expectedDuration} attribute (expressed in number of days), satisfying \textbf{P8}. 

Actor concept defines a boolean attribute called \textit{isSenior}, while \textit{TaskType} defines an additional \textit{isCritical} boolean attribute, indicating that some instance of \textit{TaskType} are flagged as critical and must be performed by senior actors. To fullfill \textbf{P9} requirement, a supplementary constraint is required for \textit{TaskType} and is captured through this invariant expressed in FML language:

\begin{lstlisting}[breaklines=true, language=java, basicstyle=\ttfamily\scriptsize, mathescape=true]
forEach (actor : allowedActors) {
    assert !isCritical | actor.isSenior
}
\end{lstlisting}

This invariant should be completed with additional constraints defined in \textit{Task} \textit{FlexoConcept}, which apply to performing actors actually assigned to enacted tasks. 

\subsubsection{Process enactment}

We now present the process enactment part of base metamodel, located right of figure \ref{fig:BaseMetamodel}. All concepts defined in this subsection are either specialization of \textit{Instance} concept (if they are associated with exactly one \textit{Type}) or \textit{MultiInstance} concept (for those which have several types).

\textit{Process} represents an enacted \textit{ProcessType}, as defined in previous subsection (\textbf{P10}). FML defines behavioural features (operations in object-oriented modelling), called \texttt{FlexoBehaviour}. \textit{ProcessType} defines following behaviour \texttt{newProcess(String)}, taking a \texttt{String} argument (the name of the process to enact):

\begin{lstlisting}[breaklines=true, language=XML, basicstyle=\ttfamily\scriptsize, mathescape=true]
public Process newProcess(String name) {    
  Process newProcess = new Process(parameters.name,this);      
  for (taskType : taskTypes) {      
    Task newTask = taskType.newTask(newProcess.name+"-"+taskType.name),newProcess);        
  }      
  return newProcess;    
}    
\end{lstlisting}

This scheme allows to rely on FML dynamic binding mechanism to delegate to types the responsability for instances creation.



% P10: Process <> ProcessType
% P11: Tasks, toutes les TaskType sont instanciées en Task
% P12: begin et end date dans Task
% P13: instantiation des Artefacts et association des actors
% P14: instantiation des Artefacts
% P15: Actor/ActorType
% P16: Artefact/ArtefactType
% P17: cf P5&P6
% P18: héritage des ActorTypes
% P19: lastUpdated dans ModelingElement

\subsection{The Acme software development process}

\subsection{Openflexo tooling}

% Expliquer l'outillage construit et comment il marche


